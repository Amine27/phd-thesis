\DontNumberAbstractPages

\begin{ThesisAbstract}
\begin{FrenchAbstract}
%\begin{spacing}{0.95}
À l'ère du \textit{Big Data}, la gestion de la consommation d'énergie par les serveurs et les centres de données est devenue un défi majeur pour les entreprises, les institutions et les pays. Parmi les applications déployées sur les centres de données, on distingue les systèmes de gestion de base de données (SGBD), qui sont l'un des principaux consommateurs d'énergie lors de l'exécution des requêtes complexes impliquant une masse de données gigantesque.
Par ailleurs, le traitement de ce type de données requiert des infrastructures logicielles et matérielles coûteuses, et qui consomment beaucoup d'énergie. Les pratiques actuelles d'utilisation et d'exploitation des bases de données extrêmement larges, montrent que le coût énergétique de traitement de requête est totalement négligé par les utilisateurs et également par les concepteurs. Sachant que le facteur le plus important pour l'utilisateur est la minimisation du temps de réponse des requêtes. Dans cette thèse nous proposons une formalisation multi-objectifs des problèmes d'exploitation des bases de données, en tenant compte de deux besoins non-fonctionnels : la performance et la consommation d'énergie lors de l'exécution d'une charge de requêtes. Cette formalisation nous a permis de tirer parti de certaines techniques avancées proposées dans l'état de l'art pour la résolution des problèmes d'optimisation multi-objectifs. Pour ce faire, nous développons en premier lieu des modèles de coût pour estimer le coût énergétique des requêtes exécutées d'une manière isolée ou parallèle. Ces modèles de coût sont ensuite intégrés dans l'un des modules les plus importants dans un SGBD, qui est le module de traitement de requêtes. La nouvelle tâche de ce module est la sélection des plans d'exécution des requêtes avec le compromis souhaité par les utilisateurs entre le temps et l'énergie des requêtes. De plus, nous proposons une initiative qui intègre la dimension énergétique dans la phase de conception physique des bases de données, afin de sélectionner des structures d'optimisation en prenant en compte les aspects énergétiques. Nous étudions le cas des vues matérialisées, l'une des structures d'optimisation redondantes très répondues. Dans chaque contribution de cette thèse, des expérimentations intensives on été menées en utilisant un dispositif réel pour les mesures d'énergie et les données des benchmarks TPC-H, TPC-DS et SBB avec plusieurs configurations matérielles et logicielles. %L'originalité de cette thèse est l'intégration du paradigme << pollueur payeur >> dans l'exploitation des bases de données.

\KeyWords{Efficacité énergétique, modèles de coût, traitement de requêtes, conception physique, gestion d'énergie, optimisation multi-objectifs.}

%\end{spacing}
\end{FrenchAbstract}
\clearpage
%--------------------------------------------------------
\includepdf[pages=-1]{include/abstract-arabic}
%--------------------------------------------------------
\begin{EnglishAbstract}
%\begin{spacing}{0.95}

In the Big Data Era, the management of energy consumption by servers and data centers has become a challenging issue for companies, institutions, and countries. In data-centric applications, Database Management Systems are one of the major energy consumers when executing complex queries involving very large databases.
Moreover, the processing of this type of data requires costly and energy-intensive computing and hardware infrastructures. Current practices in the use and exploitation of very large databases indicate that the energy cost of query is totally neglected by users and also by designers. Knowing that the most important factor for the user is minimizing the response time of queries. In this thesis we propose a multi-objective formalization of the databases exploitation techniques, taking into account two non-functional requirements: performance and energy consumption during the execution of a queries workload. This formalization allow us to take advantage of the advanced techniques proposed in the state-of-the-art for solving multi-objective optimization problems. For this purpose, we first develop cost models that estimate the energy consumption of queries executed in an isolated or parallel manner. These cost models are then integrated into one of the most important modules in a DBMS, which is the query processing module. The new objective of this module is the selection of execution plans of queries with the trade-off desired by the users between the time and the energy of the queries. Furthermore, we propose an initiative that integrates the energy dimension in the physical design phase of databases, in order to select optimization structures taking into account the energy aspects. We study the case of materialized views, one of the redundant optimization structures that is heavy used by database administrator. In each contribution of our thesis, intensive experiments are conducted using a real device for energy measurements and data of the TPC-H, TPC-DS and SBB benchmarks with various hardware and software configurations.

\KeyWords{Energy efficiency, cost models, query processing, physical design, energy management, multi-objective optimization.}

%\end{spacing}
\end{EnglishAbstract}
\end{ThesisAbstract}
