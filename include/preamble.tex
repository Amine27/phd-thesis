%%-----------------------------------------------------------------
%% Packages utilisés
%%-----------------------------------------------------------------

%\usepackage[Lenny]{fncychap} %% en-tête des chapitres
\usepackage[usenames,dvipsnames]{color} %% couleur dans le PDF
\usepackage[hyphens]{url} %% pour formatter les urls
\usepackage{amsmath,amssymb, amsthm, txfonts, pifont} %% pour les symboles mathématiques
\usepackage[L,cut=true]{style/thmbox} %% pour avoir de beaux thérorèmes, exemples ...
\usepackage{alltt} %% pour les requêtes
%\usepackage[french,nohints]{minitoc} %% pour les tables des matières de chaque chapitre
%\usepackage[boxruled, vlined, french]{algorithm2e}
%\usepackage[lined,boxed]{algorithm2e}
%\usepackage{algorithmic}
%\usepackage{float}
\usepackage[chapter]{algorithm} % FIXME: this add en empty page at the end of the document!
\usepackage{algpseudocode}

\usepackage[french]{minitoc} %% pour les tables des matières de chaque chapitre
\usepackage{multicol} %% pour faire des passages en multicolonne
\usepackage{multirow}
\usepackage{placeins, tabularx, longtable} %% pour la gestion avancée des tableaux
\usepackage{index}
\usepackage{lmodern} %% modern latin fonts
%% Utilisation d'une autre police de caractères pour le document.
%\usepackage{mathpazo}         % texte et mathématiques en Palatino
%\usepackage[sc]{mathpazo}
%\linespread{1.05}         % Palladio needs more leading (space between lines)
%\usepackage{mathptmx}         % texte et mathématiques en Times
%\usepackage{kpfonts}
%\usepackage{newpxtext,newpxmath}
%\usepackage{newtxtext,newtxmath}
%\usepackage[libertine,cmintegrals,cmbraces,vvarbb]{newtxmath}
\usepackage[mono=false]{libertine}   % text
\usepackage[libertine]{newtxmath}   % math
\usepackage[scaled=.775]{beramono}   % monospace
%\usepackage{charter,newpxmath}
%\usepackage{tgpagella}
%\usepackage{newcent,newpxmath}
%\usepackage{fourier}

\usepackage[backend=bibtex,giveninits=true,backref=true]{biblatex}
\addbibresource{include/biblio}
\usepackage{csquotes} %% pour la gestion des guillemets français
%\usepackage[babel]{microtype} %% TEST: nice text formating
\usepackage[babel,kerning]{microtype} % FIXME: kerning causes ::= to be : :=
\AtBeginEnvironment{verbatim}{\microtypesetup{kerning=false}}
\AtBeginEnvironment{listings}{\microtypesetup{kerning=false}}

%\usepackage{ae,aecompl}
\usepackage{epigraph} %% quote at the beginning of a chapter
\usepackage{etoolbox}
\usepackage[caption=false]{subfig}
\usepackage{booktabs} %% des tables professionnels

\usepackage[svgnames,table]{xcolor} %% des couleurs dans les tables
\usepackage{listings} %% pour afficher du code source
\usepackage{lscape}
\usepackage{tikz}
\usepackage{forest}
\usetikzlibrary{arrows,shapes,positioning,shadows,trees}
\AtBeginEnvironment{tikzpicture}{\catcode`\_=8} % This fix compling error

\usepackage{bookmark}
%\usepackage[pageanchor=true,draft]{style/tlhypref} %% pour avoir des hyperliens dans le pdf (draf= to solve compiling error!)
\usepackage[pageanchor=true]{style/tlhypref} %% pour avoir des hyperliens dans le pdf
\usepackage[french,nameinlink]{cleveref} %% pour ajouter le nom de chaque ref (Figure, Tableau...) (add 'capitalise' param to Capitalise)
\crefname{annex}{Annexe}{Annexes}
\crefname{paragraph}{Paragraphe}{Paragraphes}
\AtBeginDocument{\let\ref\Cref}

\usepackage[acronym,nomain,hyperfirst=false]{glossaries} % load after hyperef to make links clickable

\definecolor{lightgray}{gray}{0.9}
\definecolor{mauve}{rgb}{0.58,0,0.82}
\definecolor{dkgreen}{rgb}{0,0.6,0}

\lstloadlanguages{R,Sql,Bash}
\lstset{
  %language=R,
  %alsolanguage=sql,
  basicstyle=\small\ttfamily,       % the size of the fonts that are used for the code
  numbers=left,                   % where to put the line-numbers
  numberstyle=\tiny\color{gray},  % the style that is used for the line-numbers
  stepnumber=1,                   % the step between two line-numbers. If it's 1, each line will be numbered
  numbersep=5pt,                  % how far the line-numbers are from the code
  backgroundcolor=\color{white},  % choose the background color. You must add \usepackage{color}
  showspaces=false,               % show spaces adding particular underscores
  showstringspaces=false,         % underline spaces within strings
  showtabs=false,                 % show tabs within strings adding particular underscores
  frame=single,                   % adds a frame around the code
  rulecolor=\color{black},        % if not set, the frame-color may be changed on line-breaks within not-black text (e.g. commens (green here))
  tabsize=2,                      % sets default tabsize to 2 spaces
  captionpos=b,                   % sets the caption-position to bottom
  breaklines=true,                % sets automatic line breaking
  breakatwhitespace=false,        % sets if automatic breaks should only happen at whitespace
  %title=\lstname,                 % show the filename of files included with \lstinputlisting;
                                  % also try caption instead of title
  keywordstyle=\color{blue},      % keyword style
  commentstyle=\color{dkgreen},   % comment style
  stringstyle=\color{mauve},      % string literal style
  escapeinside={\%*}{*)},         % if you want to add a comment within your code
  morekeywords={*,...},           % if you want to add more keywords to the set
  literate={<-}{{$\gets$}}1       % gives you nicer assignment arrows
  {è}{{\`e}}1 {é}{{\'e}}1         % add all non-uft8 char here, see: https://en.wikibooks.org/wiki/LaTeX/Special_Characters#Escaped_codes
  {ê}{{\^e}}1
}

\makeatletter
\renewcommand{\ALG@name}{Algorithme}
\renewcommand{\listalgorithmname}{Liste des \ALG@name s}
\makeatother
\renewcommand{\algorithmicrequire}{\textbf{Entrée:}}
\renewcommand{\algorithmicensure}{\textbf{Sortie:}}
%\renewcommand{\algorithmiccomment}[1]{\{#1\}}
\renewcommand{\algorithmicend}{\textbf{fin}}
\renewcommand{\algorithmicif}{\textbf{si}}
\renewcommand{\algorithmicthen}{\textbf{alors}}
\renewcommand{\algorithmicelse}{\textbf{sinon}}
\renewcommand{\algorithmicfor}{\textbf{pour}}
\renewcommand{\algorithmicforall}{\textbf{pour tout}}
\renewcommand{\algorithmicdo}{\textbf{faire}}
\renewcommand{\algorithmicwhile}{\textbf{tant que}}
\renewcommand{\algorithmicrepeat}{\textbf{répéter}}
\renewcommand{\algorithmicuntil}{\textbf{jusqu'à}}
\renewcommand{\algorithmicreturn}{\textbf{retourner}}
\newcommand{\algorithmicelsif}{\algorithmicelse\ \algorithmicif}
\newcommand{\algorithmicendif}{\algorithmicend\ \algorithmicif}
\newcommand{\algorithmicendfor}{\algorithmicend\ \algorithmicfor}

\newtheorem{theorem}{Théorème}

\newcommand{\leftsemijoin}{\mbox{$\mathrel{\raise1pt\hbox{\vrule height5pt
depth0pt width0.6pt\hskip-1.5pt$>$\hskip -2.5pt$<$}}$}}

% Pour avoir le style calligraphic des symboles mathcal
%\DeclareSymbolFont{usualmathcal}{OMS}{cmsy}{m}{n}
%\DeclareSymbolFontAlphabet{\mathcal}{usualmathcal}

%\colorlet{red-black}{red!80!black}
%\colorlet{green-black}{green!70!black}
\definecolor{darkblue}{rgb}{0,0,.5}
\definecolor{greenforlinks}{rgb}{0.09, 0.45, 0.27}
% ainsi que des mots clés dans le PDF
\hypersetup{
  pdftitle={Thèse de Amine ROUKH},
  pdfauthor={Amine ROUKH},
  pdfsubject={La Prise en Compte de l'Énergie dans la Phase d'Exploitation des Bases de Données},
  pdfkeywords={Base de données, Optimisation, Énergie, Modélisation},
  hyperindex = true,
  hyperfootnotes = true,
  breaklinks = true,
  %colorlinks = false,
  bookmarks = true,
  colorlinks,
  allcolors=greenforlinks,
  linktoc=all
}
% Make the cite brackets: '[]' colored
%\DeclareCiteCommand{\cite}[\color{greenforlinks}\mkbibbrackets]
\newcommand{\mkbibbracketscol}[1]{\textcolor{greenforlinks}{\mkbibbrackets{#1}}}
\DeclareCiteCommand{\cite}[\mkbibbracketscol]
  {\usebibmacro{prenote}}
  {\usebibmacro{citeindex}%
   \usebibmacro{cite}}
  {\multicitedelim}
  {\usebibmacro{postnote}}

% \usepackage[subfigure]{tocloft}
% \renewcommand\cftchappagefont{\color{DarkGreen}}
% \renewcommand\cftsecpagefont{\color{DarkGreen}}
% \renewcommand\cftsubsecpagefont{\color{DarkGreen}}
%---SORTIES--------------------------------------------------------------------

\newif\ifpdf

\ifx\pdfoutput\undefined
   \pdffalse
\else
   \ifnum\pdfoutput=0
      \pdffalse
   \else
      \pdfoutput=1 \pdftrue
   \fi
\fi

% just for test, un-comment if nessesary!
% PDF/PS specific
% \ifpdf \pdfcompresslevel=9
% \usepackage[pdftex]{graphicx}
% \DeclareGraphicsExtensions{.jpg, .png , .pdf, .bmp} \else
% %\usepackage{graphicx}
% \usepackage[dvips]{graphicx}
% \DeclareGraphicsExtensions{.eps, .ps, .eps.gz, .ps.gz} \fi

% on pourra commenter l'utilisation de ce package pour une impression

%\usepackage{syntax} %% j'utilise ce package pour la syntaxe.
\usepackage{pdfpages} %% pour inclure la couverture

%% Il présente un bug sous miktex qui peut être corrigé par le code suivant

\makeatletter
\def\addspecial#1{%
  \remspecial{#1}%
  \expandafter\def\expandafter\dospecials\expandafter{\dospecials\do#1}%
  \expandafter\def\expandafter\@sanitize\expandafter{%
    \@sanitize\@makeother#1}%
}


\def\gr@setpar{%
  \def\par{%
    \parshape\@ne\@totalleftmargin\linewidth%
    \@@par%
    \catcode`\<12%
    \everypar{%
      \everypar{}%
      \catcode`\<\active%
      \gr@implitem\@nil%
    }%
  }%
}%


\def\gr@implitem#1<#2> #3 {%
  \sbox\z@{\hskip\labelsep\grammarlabel{#2}{#3}}%
  \strut\@@par%
  \vskip-\parskip%
  \vskip-\baselineskip%
  \hrule\@height\z@\@depth\z@\relax%
  \item[\unhbox\z@]%
  \catcode`\<\active%
}

\makeatother

%%-----------------------------------------------------------------
%% Paramétrage du formattage
%%-----------------------------------------------------------------

%% Corrections pour les imprimantes recto-verso
\ShiftOddPagesRight{-1mm}
\ShiftOddPagesDown{2.5mm}
\ShiftEvenPagesRight{0mm}
\ShiftEvenPagesDown{0mm}
\linespread{1.1}

%% Pas d'entête et de pieds de page sur les pages vides avant chapitre
\EmptyPageStyle{empty}

%% Augmentation de l'espace entre les paragraphes
\parskip=5pt

%% la longueur des captions des longtables
\setlength{\LTcapwidth}{6.3in}
%\grammarindent=4.5cm
%\grammarparsep=0.1cm

%% Pas d'espace mis automatiquement avant :
\NoAutoSpaceBeforeFDP

%% Profondeur des numérotations
\setcounter{secnumdepth}{5} %% des sections
\setcounter{tocdepth}{3} %% de la table des matières
\setcounter{minitocdepth}{2} %% minitoc de chaque chapitre

%% Ajouter une ligne après le titre de la paragraphe
\makeatletter
\renewcommand\paragraph{\@startsection{paragraph}{4}{\z@}%
   {-3.25ex\@plus -1ex \@minus -.2ex}%
   {1.5ex \@plus .2ex}%
   {\normalfont\normalsize\bfseries}}
\makeatother

%% Et pour le sous paragraphe
\makeatletter
\renewcommand\subparagraph{\@startsection{paragraph}{4}{\z@}%
   {-3.25ex\@plus -1ex \@minus -.2ex}%
   {1.5ex \@plus .2ex}%
   {\normalfont\normalsize\bfseries}}
\makeatother

% prints author names as small caps
\renewcommand{\mkbibnamegiven}[1]{\textsc{#1}}
\renewcommand{\mkbibnamefamily}[1]{\textsc{#1}}
\renewcommand{\mkbibnameprefix}[1]{\textsc{#1}}
\renewcommand{\mkbibnamesuffix}[1]{\textsc{#1}}
% Make the font size of bib small
\AtBeginBibliography{\small}

%%-----------------------------------------------------------------
%% On centre le titre des chapitres
%%-----------------------------------------------------------------

% \ChNameVar{\fontsize{14}{16}\usefont{OT1}{phv}{m}{n}\selectfont}
% \ChNumVar{\fontsize{60}{62}\usefont{OT1}{ptm}{m}{n}\selectfont}
% \ChTitleVar{\center\LARGE\bfseries\boldmath} \ChRuleWidth{1pt}

% \usepackage{pstricks}
% \makeatletter
% \def\thickhrulefill{\leavevmode \leaders \hrule height 1ex \hfill \kern \z@}
% \def\@makechapterhead#1{%
%   \reset@font
%   \parindent \z@ 
%   \vspace*{10\p@}%
%   \hbox{%
%     \vbox{%
%       \hsize=1.5cm%
%       \begin{tabular}{c}
%         \scshape \strut \@chapapp{} \\
%         %\psboxit{box 0 0 0 setrgbcolor fill}{%
%         \colorbox{black}{%
%           \vrule depth 5em width 0pt%
%           \vrule height 0pt depth 0pt width 5pt%
%           {\white \Huge \bfseries 
%             \strut \vrule height 1em depth 0pt width 0pt
%             \thechapter}%
%           \vrule height 0pt depth 0pt width 5pt%
%           }
%       \end{tabular}%
%       }%
%     \vbox{%
%       \advance\hsize by -2cm
%       \hrule height 0.4pt depth 0pt width \hsize
%       \par
%       %\vskip 6pt%
%       \vskip -6pt%
%       \hspace{20pt}%
%       \parbox{300pt}{%
%         \center \LARGE \bfseries #1}%
%       }%
%     }%
%   %\vskip 100\p@
% }
% \makeatother

\makeatletter
\usepackage[explicit]{titlesec}
\newcommand*\chapterlabel{}
\newcommand{\ifempty}[3]{\ifx#1\empty#2\else#3\fi} 
\titleformat{\chapter}
%\titleformat{name=\chapter,numberless}[display]
{\gdef\chapterlabel{} \LARGE\bfseries}
{\gdef\chapterlabel{\thechapter\ }}{0pt}
{\begin{tikzpicture}[remember picture,overlay]
\node[yshift=-10cm,xshift=3cm] at (current page.north west)
{\begin{tikzpicture}[remember picture, overlay]
%\draw (0,0.5) rectangle (1.3cm,5cm);
\fill[] (0,2) rectangle (1cm,5cm);
\draw[] (1.2cm,4.7cm) -- (15cm,4.7cm);
\node[anchor=west,yshift=4.77cm,xshift=14.75cm]{
% start flower
\begin{pgfpicture}
\pgfpathmoveto{\pgfqpoint{0.016cm}{0.027cm}}
\pgfpathlineto{\pgfqpoint{1.123cm}{0.027cm}}
\pgfpathlineto{\pgfqpoint{1.123cm}{0.812cm}}
\pgfpathlineto{\pgfqpoint{0.016cm}{0.812cm}}
\pgfpathclose
\pgfusepath{clip}
\begin{pgfscope}
\begin{pgfscope}
\pgfpathmoveto{\pgfqpoint{0cm}{0cm}}
\pgfpathlineto{\pgfqpoint{88.944cm}{0cm}}
\pgfpathlineto{\pgfqpoint{88.944cm}{88.944cm}}
\pgfpathlineto{\pgfqpoint{0cm}{88.944cm}}
\pgfpathclose
\pgfusepath{clip}
\pgfpathmoveto{\pgfqpoint{0.024cm}{0.365cm}}
\pgfpathlineto{\pgfqpoint{0.018cm}{0.336cm}}
\pgfpathcurveto{\pgfqpoint{0.035cm}{0.338cm}}{\pgfqpoint{0.058cm}{0.336cm}}{\pgfqpoint{0.071cm}{0.334cm}}
\pgfpathcurveto{\pgfqpoint{0.117cm}{0.326cm}}{\pgfqpoint{0.165cm}{0.311cm}}{\pgfqpoint{0.249cm}{0.267cm}}
\pgfpathcurveto{\pgfqpoint{0.281cm}{0.25cm}}{\pgfqpoint{0.352cm}{0.192cm}}{\pgfqpoint{0.424cm}{0.131cm}}
\pgfpathcurveto{\pgfqpoint{0.491cm}{0.075cm}}{\pgfqpoint{0.543cm}{0.028cm}}{\pgfqpoint{0.565cm}{0.029cm}}
\pgfpathcurveto{\pgfqpoint{0.592cm}{0.029cm}}{\pgfqpoint{0.601cm}{0.048cm}}{\pgfqpoint{0.606cm}{0.053cm}}
\pgfpathcurveto{\pgfqpoint{0.616cm}{0.063cm}}{\pgfqpoint{0.618cm}{0.104cm}}{\pgfqpoint{0.628cm}{0.115cm}}
\pgfpathcurveto{\pgfqpoint{0.648cm}{0.135cm}}{\pgfqpoint{0.671cm}{0.138cm}}{\pgfqpoint{0.686cm}{0.137cm}}
\pgfpathcurveto{\pgfqpoint{0.732cm}{0.133cm}}{\pgfqpoint{0.756cm}{0.112cm}}{\pgfqpoint{0.8cm}{0.099cm}}
\pgfpathcurveto{\pgfqpoint{0.839cm}{0.088cm}}{\pgfqpoint{0.866cm}{0.091cm}}{\pgfqpoint{0.899cm}{0.101cm}}
\pgfpathcurveto{\pgfqpoint{0.93cm}{0.11cm}}{\pgfqpoint{0.958cm}{0.12cm}}{\pgfqpoint{0.979cm}{0.146cm}}
\pgfpathcurveto{\pgfqpoint{1.006cm}{0.179cm}}{\pgfqpoint{0.998cm}{0.205cm}}{\pgfqpoint{1.011cm}{0.228cm}}
\pgfpathcurveto{\pgfqpoint{1.024cm}{0.252cm}}{\pgfqpoint{1.047cm}{0.276cm}}{\pgfqpoint{1.062cm}{0.299cm}}
\pgfpathcurveto{\pgfqpoint{1.079cm}{0.323cm}}{\pgfqpoint{1.108cm}{0.356cm}}{\pgfqpoint{1.121cm}{0.432cm}}
\pgfpathcurveto{\pgfqpoint{1.092cm}{0.378cm}}{\pgfqpoint{1.044cm}{0.346cm}}{\pgfqpoint{1.025cm}{0.332cm}}
\pgfpathcurveto{\pgfqpoint{0.98cm}{0.301cm}}{\pgfqpoint{0.907cm}{0.269cm}}{\pgfqpoint{0.87cm}{0.263cm}}
\pgfpathcurveto{\pgfqpoint{0.769cm}{0.247cm}}{\pgfqpoint{0.745cm}{0.247cm}}{\pgfqpoint{0.611cm}{0.248cm}}
\pgfpathcurveto{\pgfqpoint{0.488cm}{0.249cm}}{\pgfqpoint{0.409cm}{0.258cm}}{\pgfqpoint{0.379cm}{0.26cm}}
\pgfpathcurveto{\pgfqpoint{0.41cm}{0.262cm}}{\pgfqpoint{0.536cm}{0.258cm}}{\pgfqpoint{0.671cm}{0.272cm}}
\pgfpathcurveto{\pgfqpoint{0.805cm}{0.286cm}}{\pgfqpoint{0.864cm}{0.299cm}}{\pgfqpoint{0.917cm}{0.32cm}}
\pgfpathcurveto{\pgfqpoint{0.946cm}{0.331cm}}{\pgfqpoint{1.007cm}{0.362cm}}{\pgfqpoint{1.063cm}{0.424cm}}
\pgfpathcurveto{\pgfqpoint{1.082cm}{0.446cm}}{\pgfqpoint{1.102cm}{0.478cm}}{\pgfqpoint{1.114cm}{0.509cm}}
\pgfpathcurveto{\pgfqpoint{1.11cm}{0.565cm}}{\pgfqpoint{1.077cm}{0.641cm}}{\pgfqpoint{1.04cm}{0.686cm}}
\pgfpathcurveto{\pgfqpoint{0.999cm}{0.735cm}}{\pgfqpoint{0.949cm}{0.767cm}}{\pgfqpoint{0.901cm}{0.785cm}}
\pgfpathcurveto{\pgfqpoint{0.867cm}{0.798cm}}{\pgfqpoint{0.807cm}{0.811cm}}{\pgfqpoint{0.78cm}{0.81cm}}
\pgfpathcurveto{\pgfqpoint{0.797cm}{0.795cm}}{\pgfqpoint{0.84cm}{0.762cm}}{\pgfqpoint{0.894cm}{0.702cm}}
\pgfpathcurveto{\pgfqpoint{0.93cm}{0.661cm}}{\pgfqpoint{0.955cm}{0.625cm}}{\pgfqpoint{0.968cm}{0.572cm}}
\pgfpathcurveto{\pgfqpoint{0.984cm}{0.506cm}}{\pgfqpoint{0.957cm}{0.465cm}}{\pgfqpoint{0.942cm}{0.45cm}}
\pgfpathcurveto{\pgfqpoint{0.92cm}{0.429cm}}{\pgfqpoint{0.891cm}{0.421cm}}{\pgfqpoint{0.886cm}{0.426cm}}
\pgfpathcurveto{\pgfqpoint{0.876cm}{0.437cm}}{\pgfqpoint{0.881cm}{0.452cm}}{\pgfqpoint{0.88cm}{0.468cm}}
\pgfpathcurveto{\pgfqpoint{0.878cm}{0.502cm}}{\pgfqpoint{0.869cm}{0.51cm}}{\pgfqpoint{0.822cm}{0.493cm}}
\pgfpathcurveto{\pgfqpoint{0.781cm}{0.478cm}}{\pgfqpoint{0.754cm}{0.448cm}}{\pgfqpoint{0.727cm}{0.423cm}}
\pgfpathcurveto{\pgfqpoint{0.707cm}{0.404cm}}{\pgfqpoint{0.692cm}{0.389cm}}{\pgfqpoint{0.672cm}{0.378cm}}
\pgfpathcurveto{\pgfqpoint{0.655cm}{0.367cm}}{\pgfqpoint{0.621cm}{0.362cm}}{\pgfqpoint{0.616cm}{0.372cm}}
\pgfpathcurveto{\pgfqpoint{0.61cm}{0.384cm}}{\pgfqpoint{0.624cm}{0.391cm}}{\pgfqpoint{0.641cm}{0.426cm}}
\pgfpathcurveto{\pgfqpoint{0.654cm}{0.452cm}}{\pgfqpoint{0.637cm}{0.469cm}}{\pgfqpoint{0.607cm}{0.481cm}}
\pgfpathcurveto{\pgfqpoint{0.568cm}{0.496cm}}{\pgfqpoint{0.516cm}{0.482cm}}{\pgfqpoint{0.46cm}{0.447cm}}
\pgfpathcurveto{\pgfqpoint{0.401cm}{0.41cm}}{\pgfqpoint{0.388cm}{0.395cm}}{\pgfqpoint{0.346cm}{0.362cm}}
\pgfpathcurveto{\pgfqpoint{0.305cm}{0.331cm}}{\pgfqpoint{0.294cm}{0.322cm}}{\pgfqpoint{0.277cm}{0.319cm}}
\pgfpathcurveto{\pgfqpoint{0.253cm}{0.315cm}}{\pgfqpoint{0.199cm}{0.32cm}}{\pgfqpoint{0.161cm}{0.328cm}}
\pgfpathcurveto{\pgfqpoint{0.098cm}{0.343cm}}{\pgfqpoint{0.064cm}{0.352cm}}{\pgfqpoint{0.024cm}{0.365cm}}
\pgfusepath{fill}
\end{pgfscope}
\end{pgfscope}
\end{pgfpicture}
% end flower 
};
\node[anchor=north,yshift=5.65cm,xshift=0.495cm]{{{\ifempty{\chapterlabel}{}{\normalsize \sc \@chapapp{}}}}};
\node[anchor=north,yshift=4.8cm,xshift=0.49cm]{\color{white}\LARGE\chapterlabel};
\node[anchor=north west,yshift=4.5cm,xshift=1.7cm]{\begin{minipage}[t]{13cm}\vspace{0pt}\center#1\vfill\end{minipage}};
%\node[anchor=north west,yshift=4.5cm,xshift=1.7cm]{\begin{minipage}[t]{13cm}\vspace{0pt}\raggedright#1\vfill\end{minipage}};
\end{tikzpicture}};\end{tikzpicture}}
\titlespacing*{\chapter}{0pt}{50pt}{50pt}
\makeatother

%%-----------------------------------------------------------------
%% En-tête et pieds de page
%%-----------------------------------------------------------------

\setlength{\HeadRuleWidth}{0.4pt}
\pagestyle{ThesisHeadings}

%%-----------------------------------------------------------------
%% Une citation au début du chapitre
%%-----------------------------------------------------------------

%\epigraphsize{\small}% Default
\setlength\epigraphwidth{15cm}
\setlength\epigraphrule{0pt}
\renewcommand{\textflush}{flushright}

\makeatletter
\patchcmd{\epigraph}{\@epitext{#1}}{\itshape\@epitext{#1}}{}{}
\makeatother

%%---------------------------------------------------------------------------
%% Option pour la langue française
%%---------------------------------------------------------------------------
\addto\captionsfrench{%
% changer titre de la table des matière
%\renewcommand{\contentsname}{Table des matières}
%changer le titre de la liste des figures
\renewcommand{\listfigurename}{Liste des figures}
%changer le titre de la liste des tableaux
%\renewcommand{\listtablename}{Liste des tableaux}
%%changer le titre de l'abstract
%\renewcommand{\abstractname}{Abstract}
%changer la référence des tableaux
\renewcommand*{\tablename}{Tableau}
%changer la référence des figures
%\renewcommand*{\figurename}{Figure}
% Chapitre
%\renewcommand*{\chaptername}{Chapitre}
% Partie
%\renewcommand*{\partname}{Partie}
% Annexe
%\renewcommand*{\appendixname}{Annexe}
%références
%\renewcommand{\refname}{Références}
%\renewcommand{\bibname}{Références}
}


%%-----------------------------------------------------------------
%% Liste de commandes utilisées dans la thèse
%%-----------------------------------------------------------------

\newcommand{\vs}{\vspace{1ex}}              % Small vertical space

\newcommand\code[1]{{\tt #1}}
\def\<#1>{\synt{#1}}

\newcommand\squote{\textrm{\code{'}}}
\newcommand\diese{\textrm{\code{\#}}}

\newcommand\spec[1]{\textrm{\code{#1}}}

\newcommand\total{$\bullet$}

\newcommand\partiel{$\circ$}

\newcommand\aucun{-}

% Pour la numérotation des équations
\renewcommand{\theequation}{\thesection \Alph{equation}}


\newcommand\codeMath[1]{{$\mathtt{#1}$}}
\newcommand\Font[1]{{\footnotesize \mathtt{#1}}}

%\newtheorem[S,underline=false]{example}{Exemple}{\itshape}{\rmfamily}
%\newtheorem{example}{Exemple}{\itshape}{\rmfamily}
% \renewenvironment{example}[1][Exemple]{%
%   \par%
%   \list{\hspace\labelsep\textit{#1.}}{%
%     \leftmargin=\parindent%
%     \labelwidth=\parindent}%
%   \item\relax}{%
%   \endlist}

\newcounter{example}[chapter]
\renewenvironment{example}[1][]{\refstepcounter{example}\par\medskip
   \noindent \textbf{Exemple~\theexample. #1} \itshape\rmfamily}{\medskip}

\newenvironment{explication}[1][Explication]{%
  \par%
  \list{\hspace\labelsep\textit{#1.}}{%
    \leftmargin=\parindent%
    \labelwidth=\parindent}%
  \item\relax}{%
  \endlist}

  \newenvironment{remarque}[1][Remarque]{%
  \par%
  \list{\hspace\labelsep\textit{#1.}}{%
    \leftmargin=\parindent%
    \labelwidth=\parindent}%
  \item\relax}{%
  \endlist}

%\newtheorem[S,underline=false]{usecase}{Use case}{\itshape}{\rmfamily}

\newtheorem{definition}{Définition}

\newtheorem[L,cut=false]{exigence}{Exigence}

\newtheorem{contrainte}{Contrainte}

%% Commandes pour saisir des noms

%\newcommand{\encore}{\mbox{\em Encore }}    % Algèbre Encore
%\newcommand{\ontoalgebra}{\mbox{\em OntoAlgebra }}    % Algèbre OntoAlgebra

\newenvironment{abstract}{%
      \list{}{\advance\topsep by0.35cm\relax
      \leftmargin=1cm
      \labelwidth=1cm
      \listparindent=1cm
      \itemindent\listparindent
      \rightmargin\leftmargin}\item[\hskip\labelsep
                                    \bfseries Résumé.]}
    {\endlist}

\newenvironment{monitemize}%
{ \begin{list}%
        {--}%
        {\setlength{\labelwidth}{30pt}%
         \setlength{\leftmargin}{30pt}%
         \setlength{\topsep}{0pt}%
         \setlength{\itemsep}{2pt}}}%
{ \end{list} }

\newcounter{enum}
\newenvironment{monenumerate}
{ \begin{list}%
    {\arabic{enum}. }{
    \usecounter{enum}%
         \setlength{\labelwidth}{30pt}%
         \setlength{\leftmargin}{30pt}%
         \setlength{\topsep}{0pt}%
         \setlength{\itemsep}{2pt}}}%
{ \end{list} }

\newcounter{hypcpt}
\newenvironment{hyp}
{ \begin{list}%
    {\uline{Hypothèse \arabic{hypcpt}.} }{
    \usecounter{hypcpt}%
         \setlength{\labelwidth}{30pt}%
         \setlength{\leftmargin}{30pt}%
         \setlength{\topsep}{0pt}%
         \setlength{\itemsep}{2pt}}}%
{ \end{list} }

\newcounter{enumpar}
\newenvironment{monenumeratepar}
{ \begin{list}%
    {(\arabic{enumpar}) }{
    \usecounter{enumpar}%
         \setlength{\labelwidth}{30pt}%
         \setlength{\leftmargin}{40pt}%
         \setlength{\topsep}{0pt}%
         \setlength{\itemsep}{\parsep}}}%
{ \end{list} }

% Par défaut la commande \cite ne coupe pas les mots
% ce qui provoque le débordement dans la marge
% voici le code permettant de corriger ce comportement

\makeatletter
\def\@citex[#1]#2{%
  \let\@citea\@empty
  \@cite{\@for\@citeb:=#2\do
    {\@citea\def\@citea{,\penalty\@m\ }%
     \edef\@citeb{\expandafter\@firstofone\@citeb\@empty}%
     \if@filesw\immediate\write\@auxout{\string\citation{\@citeb}}\fi
     \@ifundefined{b@\@citeb}{\mbox{\reset@font\bfseries ?}%
       \G@refundefinedtrue
       \@latex@warning
         {Citation `\@citeb' on page \thepage \space undefined}}%
       {{\csname b@\@citeb\endcsname}}}}{#1}}
\makeatother
 
