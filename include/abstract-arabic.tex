%%!TEX TS-program = pdflatex
%
% \documentclass[a4paper,10pt,final]{book}
% \usepackage[utf8]{inputenc}
% \usepackage[T1]{fontenc}
% \usepackage{microtype}
% \usepackage{geometry}
%  \geometry{
%  a4paper,
%  total={170mm,257mm},
%  left=20mm,
%  right=30mm,
%  top=30mm,
%  }
% \usepackage{arabtex} 
% \usepackage{utf8}
% 
% \pagenumbering{gobble}
% 
% \begin{document}
% \setcode{utf8}
% \section*{\centering\RL{ملخص}}
% %\DontFrameThisInToc
% %\addcontentsline{toc}{tlstarchapter}{\RL{ملخص}}
% %\addcontentsline{toc}{spsection}{\RL{ملخص}}
% \begin{RLtext}
% وعلاوة على ذلك، وعلاج هذا النوع قاعدة بيانات تتطلب تكنولوجيا باهظة الثمن والبنية التحتية والطاقة المركزة. وتبين الممارسات الحالية لاستخدام واستغلال  (قواعد بيانات كبيرة للغاية) أن تكلفة الطاقة من معالجة الاستعلام مهملة تماما من قبل المستخدمين، وكذلك من قبل المصممين. مع العلم أن أهم عامل للمرة استجابة استعلام المستخدم، واحدة من الأهداف النهائية من هذه الرسالة هو لتطوير نماذج التكلفة لتصميم الجزء المادي من اختيار الهياكل التحسين مثل مؤشر ، التقسيم، تتحقق وجهات النظر. هذا الاختيار يجب أن تأخذ في جوانب الطاقة الحساب. أصالة هذا العمل هو دمج النموذج ``الملوث يدفع'' في استخدام قواعد البيانات. تحديد نماذج التكلفة، ودمج كل من وقت الاستجابة (مستوى المهارة العالية للفريق البيانات الهندسية والنماذج المختبرية \LR{LIAS-ENSMA}) وتكاليف الطاقة، بسهولة يقيس نوعية التقنية الأمثل.
% \par\bigskip\noindent
% \textbf{الكلمات المفتاحية\LR{\textbf{:}}} التصميم المادي، معالجة الاستعلام، نماذج التكلفة، اختبار مقاعد البدلاء، تقييم الأداء، كفاءة الطاقة.\par
% \end{RLtext}
%  \end{document}


%!TEX TS-program = xelatex
\documentclass[a4paper]{book}
\usepackage{microtype}
\usepackage{indentfirst}
\usepackage{geometry}
 \geometry{
 a4paper,
 total={170mm,257mm},
 left=20mm,
 right=30mm,
 %top=30mm,
 }
\usepackage{setspace}
\usepackage{fontspec}
\usepackage{polyglossia}
\setmainlanguage{arabic}
\setotherlanguage{english}
\setmainfont{Adobe Naskh Medium} % Adobe Arabic
%\newfontfamily\arabicfont[Script = Arabic,SizeFeatures={Size=16}]{Adobe Naskh Medium} % Adobe Arabic, Adobe Naskh Medium

\pagenumbering{gobble}

\begin{document}
%\setstretch{1.4}

\chapter*{\centering{مُلَخَّصٌ}}

{\fontsize{14pt}{16pt}\selectfont في عصْرِ  \textit{البيَاناتِ الضخمةِ،} أصبحت إدارَةُ استهلاك الطَّاقة بالنِّسبَة للخَوَادِمِ ومراكز البيانات تحدِّيًا كبيرا للشركات والمؤسسات والدُّوَل. من بين التطبيقات المنتشرة بكثرة على مراكز البيانات، نَجِدُ أنَّ أنظمة إدارة قواعد البيانات واحدة من أكبر مستهلكي الطَّاقة  الكهرَبائيَّة، وذلك أثناء تنفيذ الاسْتِعْلاَمَات المعقَّدة التي تنْطوي على حجمٍ كبير جدًّا من البيانات.
وعِلاوةً على ذلك، عِلاجُ هذا النوع من قواعد البيانات يتطلَّب تكنولوجيا وبُنيَة تحتية باهظةَ الثَّمن ومستهلكةً كبيرةً للطَّاقة الكهربائيَّة. إلى زمن غيْرِ بعيد، كانت تَكْلُفَةُ الطَّاقة أثناء معالجة الاستعلامات الخاصَّة باستخدام وتشغيل قواعد البيانات كبيرةِ الحجم مُهْمَلَةً تماما، سواءا من قِبَلِ المستخدمين أو من قبل المُصمِّمين. حيث أنَّ العامل الأكثَرَ أهمِّية للمستخدِم هو تقْليلُ زمَن مُعالَجة الاستعلامات وتَلَقِّي النَّتائج بسُرعة. في هذه الأُطروحة نقترحُ صِياغةً مُتعدِدة الأهداف لمشاكل استغلال قواعد البيانات، وهذا عن طريق الأخْذِ بعيْن الاعتبار كلاًّ من الاحتياجات غيْرِ الوظِيفِيَّة: تحسينُ الأداءِ وتقليلُ استهلاك الطَّاقة عند تشغيل مجموعةٍ من الاستعلامات. هذه الصِّياغةُ سَمَحَت بالاستفادة من التِّقنيَّات المُتقدِّمة المُقترَحة في حالة التِّقنيِّة الصِناعيَّة السابقة من أجل حلِّ مُشكل الأَمْثَلَةِ مُتعدِّدَةِ الأهداف. لهذا، في أوَّلِ الأمر قُمنا بتطوير نماذجِ التَّكلُفَة لتقدير تكلُفة الطَّاقة اللاَّزمة لتشغيل الاستعلامات، بطريقة معزولةٍ أو متوازيةٍ. بعد ذلك، قمنا بدَمج هذه النَّماذج في واحِدَة من أهمِّ الوحدات في نُظم إدارة قواعد البيانات، ألا وهي وِحدة معالجة الاستعلام. الهدف الجديد لهذه الوِحدة هو اختيارُ خُطَطِ تنفيذ الاستعلام مع الأخذ بعيْن الاعتبار لمجموعةٍ من قِيَمِ التَّسْوِية والمُقايضة بيْن وقت التشغيل والطَّاقة المُستهلكة، هذه القِيمُ المستخدمون هُمُ المسؤولون عن إدخالها. أيضا، اقترحنا صيغَةً لدمجِ البُعدِ الطَّاقويِّ في مرحلة التصميم الماديِّ لقواعد البيانات، وذلك عن طريق اختيار هيَاكلِ الأَمْثَلَةِ مع الأخذ بعين الاعتبار جانِبَ استهلاك الطَّاقة. نتيجة لذلك، قمنا بدراسة تقنيِّة حِفظ نتائج الاستعلامات، والتي تُعدُّ واحدةً من أهمِّ هياكِل الأمثَلة المُستعملة بكثرة. في كلِّ مساهمة من مساهمات هذه الأُطروحةِ، قمنا بإجراء تجاربَ واسعةِ النِّطاق باستخدام جهاز فعليٍّ لقياس استهلاك القُدرة الكهربائيَّة للخادِم، واعتمادًا على البيانات والاستعلامات الخاصَّة بمقاييسِ كلٍّ من ،TPC-DS TPC-H وSBB مع الحِرص على تنْويع المكوِّنات المادِّية والبرمجيَّة في كلِّ تَجربة.

\par\bigskip\noindent
{\addfontfeatures{FakeBold=1.5}الكلمات المفتاحيَّة:} \addfontfeatures{FakeBold=0} كفاءةُ استخدام الطَّاقة، نماذجُ التَّكلفة، معالَجةُ الاستعلام، التَّصميم الماديُّ، إدارة الطَّاقة، أمثَلةٌ متعدِّدة الأهداف.\par}

\end{document}