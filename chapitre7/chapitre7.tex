\chapter{Conclusion générale et Perspectives}
\label{chap7}

\epigraph{<< Research is to see what everybody else has seen, and to think what nobody else has thought. >>}{--- \textup{Albert Szent-Gyorgyi}}

\NoChapterPrefix \NoChapterNumberInRef {\hypersetup{linkcolor=black} \minitoc}
 
%% numérotation des figures, des tables et des équations préfixé par le numéro de chapitre
\makeatletter
\renewcommand{\thefigure}{\ifnum \c@section>\z@ \thechapter.\fi
 \@arabic\c@figure}
\@addtoreset{figure}{chapter}
\makeatother

\makeatletter
\renewcommand{\thetable}{\ifnum \c@section>\z@ \thechapter.\fi
 \@arabic\c@table}
\@addtoreset{table}{chapter}
\makeatother

\makeatletter
\renewcommand{\theequation}{\ifnum \c@section>\z@ \thechapter.\fi
 \@arabic\c@equation}
\@addtoreset{equation}{chapter}
\makeatother
%%-----------------------------------------------------------------
%% Résumé
%%-----------------------------------------------------------------
 
%laisser une page ou deux pages vides, telle est la question !
%\EmptyNewPage
\newpage
 
%********************************************************************************
%********************************************************************************
\section{Conclusion}

Nous présentons dans ce chapitre un bilan du travail que nous avons effectué ainsi qu'un ensemble d'ouvertures et de perspectives pour ce travail.

À l'heure actuelle, la plupart des gens sont conscients de la gravité du problème de l'énergie : nos sources d'énergie primaires sont en train de s'écouler, tandis que la demande d'énergie dans les environnements commerciaux et domestiques ne cesse d'augmenter. En outre, les effets secondaires de la consommation d'énergie ont d'importantes considérations environnementales à l'échelle mondiale. L'émission de gaz à effet de serre, comme le $CO_2$, que la plupart des climatologues considèrent aujourd'hui comme étant liée au réchauffement climatique, n'en est qu'un exemple. Les scientifiques et les politiciens du monde mettent l'accent sur une solution stratégique : Le développement de nouvelles sources d'énergies propres et renouvelables est nécessaire et vital pour surmonter le problème énergétique. Steve Chu, l'ancien secrétaire américain à l'énergie, a placé cette situation dans son contexte \cite{Chu08} : << Une double stratégie est nécessaire pour résoudre le problème énergétique : (1) maximiser l'efficacité énergétique et diminuer la consommation d'énergie ; (2) développer de nouvelles sources d'énergies propres. La stratégie (1) restera la plus simple à réaliser pour les prochaines décennies. >>

Le facteur clé à considérer avec les systèmes informatiques d'aujourd'hui est que la quantité de puissance électrique qu'ils consomment ne s'ajuste pas en fonction de la quantité de travail que le système réalise. Jusqu'à présent, le principal objectif de la plupart des systèmes informatiques à usage général était de maximiser les performances sans tenir compte de la consommation d'énergie. Cela évolue rapidement à mesure que nous approchons du point où le coût d'acquérir un équipement informatique sera dépassé par le coût de l'énergie pour le faire fonctionner, sauf si nous accordons une importance à la conception de systèmes qui consomment moins d'énergie. Cependant, la puissance et l'énergie ont commencé à restreindre sérieusement la conception des équipements, des systèmes, des clusters, des centres de données et des applications. Une meilleure conception de l'équipement et de meilleures politiques de gestion de l'énergie sont nécessaires pour répondre à ces préoccupations.

Nous nous sommes intéressés dans cette thèse au problème de l'intégration de l'énergie dans les bases de données avancées afin de minimiser leur consommation. Les BDs sont aux cœur des centres de données, et elles sont responsables de la majeure partie de leurs consommations énergétiques. Notre solution proposée pour diminuer la consommation d'énergie des BDs, est de formaliser le problème comme un problème d'optimisation. Dans cette formalisation, il faut ajuster les ressources logicielles afin de minimiser leurs consommation en gardant un certain niveau de performance désirée.
En analysant l'état de l'art et à notre connaissance, nous avons constaté qu'il n'existait pas de démarche pour intégrer l'énergie dans les bases de données. En se basant sur les travaux existants, et d'après notre étude de ce problème, nous avons proposé au cours des chapitres de cette thèse des lignes directrices détaillées qui établissent un référentiel à la communauté des bases de données pour concevoir des BDs vertes. Les points essentiels de cette démarche sont : (1) la proposition des modèles pour quantifier la consommation des taches exécutées par un SGBD, (2) le développement des techniques pour résoudre le problème d'optimisation d'énergie sans sacrifier la performance et (3) l'intégration de ces modèles et techniques dans le fonctionnement interne des BD et SGBD.

\subsection{État de l'art}
Notre travail a été guidé par une étude préalable des travaux existants à fin d'examiner les dimensions principales qu'ils faut prendre en charge lors de l'intégration de l'énergie. Nous avons divisé notre étude sur trois axes : le premier se focalise sur les problèmes de conception et de traitement des bases de données. Nous avons étudié le cycle de vie d'une base de données avec un accent sur les structures d'optimisation dans la phase de conception physique. Nous avons également décrit le processus du traitement de requêtes dans un SGBD relationnel. Le deuxième axe traite le problème d'optimisation multi-objectifs. Nous avons présenté le problème dans sa forme générale avec les techniques de résolution. Ensuite, nous avons étudié les travaux sur les problèmes multi-objectifs dans les bases de données. En fin, le troisième axe concerne l'efficacité énergétique dans les systèmes informatiques. Nous avons passé en revue les travaux visant à minimiser l'énergie au niveau matériel, système d'exploitation et application. Nous avons détaillé les études sur ce dernier qui comprend les bases de données. Nous avons identifié les lacunes des approches existantes et nous avons proposé nos démarches pour les améliorer.
 
\subsection{Modèle de coût énergétique}
Notre première contribution est la proposition d'un modèle de coût pour estimer la consommation d'énergie lors de l'exécution d'un ensemble de requêtes. Nous avons indiqué comment la modélisation au niveau des pipelines pourrait être un indicateur robuste de la prédiction d'énergie. Pour garantir la portabilité de notre approche, nous avons choisi les coûts d'E/S et CPU des pipelines comme paramètres de notre modèle. Alors qu'un certain nombre d'études récentes ont exploré ce problème, la majorité des travaux existants considère la prédiction de l'énergie pour une seule requête isolée. Dans cette thèse, nous avons également considéré le problème le plus général avec plusieurs requêtes concurrentes. Ceci est utile pour les tâches de la gestion de nombreuses bases de données, y compris le contrôle d'admission, l'ordonnancement des requêtes et le contrôle d'exécution avec l'efficacité énergétique comme un objectif primaire.
Le modèle de coût est construit sur la base de résultats empiriques obtenus à partir d'une phase d'apprentissage des charges de requêtes qui ont été soigneusement créés. Les paramètres sont calculés par un algorithme de régression polynomiale multiple. De plus, nous avons effectué des tests sur notre framework avec des bases de données réelles, tel que SSB, TPC-H et TPC-DS, en exécutant un ensemble de requêtes et en comparant leurs coûts d'énergie avec ceux prédits par notre modèle de prédiction. Nos résultats montrent que le modèle peut prédire l'énergie avec une grande précision.

\subsection{Compromis entre la performance et l'énergie}
À fin d'incorporer l'énergie dans les bases de données, il est essentiel de choisir un compromis entre l'efficacité énergétique et la dégradation en performance. Contrairement au cas d'optimisation mono-objectif où la solution optimale est généralement unique, dans notre cas d'optimisation multi-objectif, il existe plutôt un ensemble de solutions de compromis alternatives. En analysant l'état de l'art, la plupart des travaux ignorent ce choix lors de l'intégration de l'implémentation de leurs approches.
Dans cette thèse, nous avons proposé et développé des techniques d'optimisation multi-objectifs qui offrent le meilleur compromis entre le temps de réponse aux requêtes et la consommation d'énergie du système. Notamment, nous avons utilisé deux techniques, une classique connue sous le nom de la méthode sommes pondérées, et la deuxième technique est basée sur les algorithmes évolutionnaires, connue sous le nom de NSGA-II. Ces deux techniques ont été utilisées d'une façon a posteriori, où nous donnons le choix aux administrateurs de bases de données et aux utilisateurs de choisir le meilleur compromis parmi une liste de frontière de Pareto.
 
\subsection{L'énergie dans le traitement de requêtes}
Jusqu'à présent, la conception du système de base de données s'est concentrée sur l'amélioration de la performance lors de la phase de traitement des requêtes. Nous avons exploré le potentiel de la conservation de l'énergie dans les systèmes de gestion des bases de données relationnelles, en étudiant chaque étape du processus de traitement des requêtes. Notre deuxième contribution consiste en la modification d'optimiseur de requêtes dans un SGBD pour prendre en compte le coût énergétique des plans de requêtes. À cette fin, nous avons pris le cas du SGBD open-source PostgreSQL, son module de traitement de requêtes a été étudié et son code a été modifié pour inclure la dimension d'énergie. Un outil, nommé EnerQuery, a été développé pour permettre aux DBA et aux utilisateurs de diagnostiquer le comportement de la consommation d'énergie des systèmes et des BDs. Cet outil permet aussi de visualiser le plan de requêtes avec les segmentations de pipelines correspondantes et diverses informations de coût pour chaque opérateur, il permet également de définir des valeurs de compromis entre l'énergie et le temps d'exécution. Les résultats d'expérimentations affirment que les techniques proposées peuvent réduire la consommation d'énergie des serveurs de base de données et contrôler les compromis entre la consommation et la performance du système en choisissant des plans d'exécution alternatifs.

\subsection{L'énergie dans la conception physique des BD}
Comme troisième contribution, nous avons proposé une initiative visant à intégrer l'énergie dans la phase de conception physique du cycle de vie des bases de données, en considérant le cas des vues matérialisées - une structure d'optimisation redondante.
Avant de présenter notre méthodologie pour résoudre le problème de la sélection des vues matérialisées, nous avons analysé et discuté les différents scénarios d'intégration de l'énergie dans le processus de sélection des vues. Dans cette thèse, nous nous sommes concentrés sur deux scénarios : l'énergie comme un besoin non-fonctionnel et l'énergie comme contrainte. Ces deux scénarios ont été formalisés et résolus à l'aide d'algorithmes évolutionnaires multi-objectifs. Pour évaluer la qualité des solutions finales, nous avons développé des modèles de coûts mathématiques pour estimer les coûts des objectifs de notre problème : la consommation d'énergie et la performance des requêtes. Les résultats d'évaluation de notre proposition ont été confrontés à ceux obtenus par la validation réelle avec un matériel de mesure d'énergie.
En outre, nous avons souligné l'importance de la conception physique de la base de données vers les SGBD à haut rendement énergétique. L'étude expérimentale confirme notre affirmation car l'approche proposée réduit de manière significative la consommation d'énergie globale de la charge de requêtes, en permettant des économies de puissance active jusqu'à 38\% et des économies d'énergie totale jusqu'à 84\%.
En outre, nous avons étudié un autre facteur important dans les environnements cloud d'aujourd'hui qui vise à réduire la consommation d'énergie tout en répondant à un seuil de performance sous un contrat de SLA.

\section{Perspectives}
Les travaux initiés dans cette thèse peuvent se poursuivre dans de nombreuses directions. Nous esquissons ici quelques pistes de perspectives à court et à long terme.

\subsection{Extension du modèle de coût}
% + reg, perfo-counter...
Nous avons proposé un modèle de coût énergétique basé sur les paramètres d'E/S et de CPU des pipelines et la régression polynomiale. En premier lieu, il serait intéressant d'augmenter le modèle avec d'autres paramètres lié au matériels, tel que les compteurs de performance \cite{Contreras05}. Ces derniers donnent l'accès aux différents types d'événements générés sur le processeur et la mémoire, ces événements permettent de comprendre le comportement d'énergie et les types d'actions qui influencent sur la consommation du système. Également, il est important de comprendre comment développer des modèles précis pour les autres composants du système, comme les équipements de réseau et la carte graphique. En deuxième lieu, l'utilisation des techniques d'apprentissage automatique avancées, telles que la méthode d'analyse canonique des corrélations \cite{Ganapathi09}, Gradient boosting \cite{Li12a,Konig11}, machines à vecteurs de support \cite{Akdere12} et la méthode de lasso \cite{Zheng15} est l'une des directions pour des futures recherches. Ces techniques sont capables de donner des résultats d'estimation avec une grande précision, mais la complexité de la modélisation et le temps de calcul sont les principaux inconvénients de ces méthodes. 
 
\subsection{Généralisation de la formalisation multi-objectifs avec d'autres structures d'optimisation}
% bitmap index, 
Dans notre thèse, nous avons pris le problème de sélection des vues matérialisées comme un cas d'étude sur l'intégration de l'énergie dans la phase de conception physique des bases de données. Ainsi, nous avons proposé une formalisation multi-objectifs pour résoudre ce problème. Cette formalisation peut être généralisée pour d'autres structures d'optimisation, comme le problème de sélection d'index binaires, ou le problème de fragmentation, du fait que le processus de recherches des données à partir des indexes binaires peut avoir un impact important sur la consommation d'énergie du système. De plus, notre formalisation peut être combinée avec d'autres besoins non-fonctionnels, tel que le coût monétaire dans les environnements de cloud, on auras en résultats plus de deux $\mathcal{BNF}$. En outre, dans ce cas, nous pouvons exploiter les techniques émergentes des algorithmes évolutionnaire scalables nommés : optimisation << many-objectifs >> \cite{Chand15}.

\subsection{Le cas du traitement de requêtes parallèles ou distribuées}
% distribued query processing
% exploit our batch of query power model
Nous avons étudié le problème d'efficacité énergétique dans le contexte de bases de données centralisées. Il serait judicieux d'adapter les approches proposées dans le cadre de base de données parallèles ou distribuées particulièrement pour la phase de traitement de requêtes. Dans ce contexte, il faut adapter, en première étape, le modèle de coût avec les caractéristiques du réseau à fin d'estimer le coût de transfert des données entre les sites. En deuxième étape, il faut développer des approches pour étendre le module de traitement de requêtes parallèles ou distribuées avec la dimension d'énergie, la formalisation du nouveau problème sera : choisir le meilleur plan d'exécution des requêtes qui minimise à la fois les coûts du temps de réponse, du transfère réseau et de la consommation d'énergie. Pour avoir ce plan, il faut employer des techniques avancées lors de la phase de génération et comparaison des plans, tels que les algorithmes évolutionnaires ou bien les algorithmes aléatoires \cite{Trummer16b}. De plus, dans le cas des BDs parallèles ou distribuées, nous pouvons exploiter notre modèle de coût pour les requêtes concurrentes comme dans \cite{Psaroudakis13}, chose qui n'est pas abordée dans cette thèse.

 
\subsection{L'ordonnancement de requêtes et la gestion du cache}
% + feedback control mechanism
Une technique intéressante pour minimiser l'énergie dans une base de données est l'ordonnancement de requêtes. Cette technique est très utile dans les environnements de cloud Database-as-a-Service (DBaaS) ou il faut avoir une politique de contrôle d'admission de requêtes \cite{Xiong11,Chi11,Chi13}. Dans une charge de requêtes, souvent il existe des requêtes qui partagent des sous-expressions et de données communes. Cependant, au lieu d'exécuter les requêtes d'une manière séquentielle ou aléatoire, il faut proposer des méthodes pour planifier l'ordre d'exécution et sauvegarder les résultats intermédiaires des opérations tout en exploitant les parties communes des requêtes, afin de minimiser le temps d'exécution et la dissipation d'énergie. La sauvegarde de résultats intermédiaires est généralement sous la contrainte de la taille de mémoire ou de disque. Ainsi, il existe deux problèmes : d'abord le problème de trouver le meilleur ordre d'évaluation et d'exécution des expressions, appelé le problème \textit{d'ordonnancement des requêtes}, et le problème de décider à quel moment admettre la sauvegarde d'un résultat commun dans la mémoire cache, et quand le retirer, appelé le problème de \textit{gestion du cache}. La méthode courante pour résoudre ces problèmes est la méthode d'optimisation multi-requêtes \cite{Sellis88,Gupta01,Diwan06}, il faut revoir les algorithmes basés sur cette méthode pour rajouter la dimension d'énergie.

\subsection{Proposition des approches orientées matériel}
% DVFS, Co-Processing CPU/GPU
Le travail réalisé au cours de cette thèse a principalement consisté à élaborer des techniques d'amélioration d'efficacité énergétique orientées logicielles. Toutefois, la deuxième partie des techniques, orientées matériel, est connue pour avoir des résultats de minimisation d'énergie très intéressants grâce à la propriété d'énergie proportionnel du matériel \cite{Skadron04}. Par contre, nous proposons de combiner les deux branches de solutions dans le cas des bases de données. Par exemple, au lieu d'utiliser la technique de DVFS pour changer la fréquence du CPU d'une manière isolée, il serait préférable de développer une méthode qui permet d'ajuster les valeurs de DVFS suivant la charge de travail actuel directement à partir du SGBD. Le cas le plus évident pour introduire cette méthode est le traitement des requêtes, une EE significative peut être achevée si nous pouvons trouver des plans d'exécution qui ne nécessitent pas autant de temps CPU. Pour les équipements de stockage multi-disque, au lieu de se focaliser uniquement sur le problème \textit{d'équilibrage de charge} pour améliorer les performances d'E/S. Cependant dans notre cas, nous proposons de développer une technique de \textit{consolidation de charge} en plaçant les charges d'E/S les plus accédées dans les mêmes portions de disques \cite{Srikantaiah08}. De cette façon, des possibilités d'économie d'énergie sont créées car les disques à faible charge d'E/S peuvent basculer vers des modes de faible consommation/performance.
 
\subsection{Benchmark pour l'énergie}
% car nous avons crée nos workloads manuellement
Dans la phase d'apprentissage lors de la création de notre modèle de coût énergétique, nous avons opté pour crée nos propres charges de requêtes manuellement pour collecter les paramètres d'E/S, de CPU et les mesures d'énergie à fin d'appliquer la méthode de régression. Ce processus était le même pour les travaux précédents connexes \cite{Xu13,Kunjir12,Lang11,Rodriguez11}, car il n'existe pas un benchmark énergétique standard avec des requêtes SQL pour examiner le comportement d'énergie des composantes matériels. L'approche classique est soit de crée des charges de requêtes personnalisées soit d'utiliser des outils séparés (comme Phoronix Test Suite\footnote{\url{http://www.phoronix-test-suite.com/}}) pour examiner l'ensemble d'équipements matériels. Les deux approches souffrent des problèmes de portabilités entre les systèmes, et ne garantissent pas un bon examen des équipements. Cependant, il serait très bénéfique de développer un benchmark avec des requêtes SQL caractérisées par des opérations gourmandes en CPU, par des opérations exhaustives en terme de ressources de stockage (gourmandes en E/S) et par des requêtes avec un taux élevé de transfère réseau.
